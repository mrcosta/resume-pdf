%!TEX TS-program = xelatex
\documentclass[]{friggeri-cv}
\usepackage{multicol}

\begin{document}
\header{mateus}{costa1}
       {consultor/desenvolvedor}

% In the aside, each new line forces a line break
\begin{aside}
  \section{sobre}
    Belo Horizonte
    27 anos
  \section{mail}
    mateuscomp\\@gmail.com 
  \section{skype}
    mateuscomp
  \section{github}
    \href{https://github.com/mrcosta}{mrcosta}
  \section{linkedin}
    \href{www.linkedin.com/in/mateusrodriguescosta}{mateus costa}
  \section{línguas}
    inglês fluente e espanhol iniciante
  \section{habilidades pessoais}
  curioso
  proativo
  bom ouvinte
  trabalho em grupo
  compartilhador
\end{aside}

\section{Quem eu sou:}

Motivado, curioso e proativo desenvolvedor com paixão para construir software com soluções criativas e que traga valor para quem vá usar. Tenho trabalhado nos últimos anos com boas práticas de desenvolvimento e tecnologias atuais. Leitor assíduo de livros e blogs, para aprender e entender novas tecnologias que possam me ajudar no dia-a-dia.\\
    
    Também tento trabalhar em projetos paralelos, que tragam valor para a comunidade, com o intuito também de testar novas tecnlogias.

\section{Onde estudei:}

\begin{entrylist}
  \entry
    {2012 – 2014}
    %{Ph.D. {\normalfont candidate in Computer Science}}
    {Mestrado}
    {\textbf{CEFET-MG}}
    {\textbf{Curso:} Modelagem Matemática e Computacional\\ \textbf{Linha de estudo:} Modelagem e Otimização de Processos}
  \entry
    {2008 – 2011}
    {Graduação}
    {\textbf{UNIFOR-MG}}
    {\textbf{Curso:} Ciência da Computação}  
\end{entrylist}

\section{Trabalho/Trabalhei:}
\begin{entrylist}
  \entry
    {2014 - agora}
    {{\normalfont Atuo na \textbf{Thoughtworks} como consultor e desenvolvedor em diferentes projetos. O trabalho de consultoria envolve entender o negócio do cliente e como podemos ajudá-lo a resolver problemas com metodologias ágeis e processos de desenvolvimento de software que o ajudem a ter resultados mais rápidos e satisfatórios. Alguns projetos que participei:\\
    \begin{itemize}  
\item API para prover serviços de check-in de vôos para uma companhia aérea sul-americana.
\item Projeto de gerenciamento de conteúdo de um site de notícias onde a segurança das informações, processos e pessoas envolvidas no produto, era prioridade.
\item API para prover serviços de compras de passagens empresarias para uma companhia aérea americana.
\item Dashboard desenvolvido para uma companhia aérea americana para acompanhar o status de Turn (operações realizadas para um avião pousar e decolar novamente) de seus vôos.
\end{itemize}
\textbf{Algumas Tecnologias:} Java, Python, Spring, SpringBoot, Flask, JUnit, Spock, Cucumber, Wiremock, RestAssured, Gradle, Maven, Durandal, FlightJS, Saas, Docker, LambdaCI, Ansible
    }}{}{}
  
  \entry
    {2013 - 2014}
    {{\normalfont \textbf{FITec} como Engenheiro de Software.  Participei de um time distribuiído em 3 países para migrar um software existente, que lida com processamento de vídeos para a AWS (Amazon Web Services). Também criamos uma aplicação em Grails para esse sistema com várias novas funcionalidades.
        \\\textbf{Algumas Tecnologias:} Grails, Groovy, Spock, Jasmine, Grunt, Saas, Jasmine, Angular e AWS (PaaS)
    }}{}{}
\end{entrylist}

\section{Principais linguagens de programação que domino:}
Java (4 anos)\\
Groovy (1,5 anos)\\
Python e Perl (1 ano)\\

\section{Algumas práticas e metodologias que trabalho:}
\begin{multicols}{2}
    TDD e Testes Automatizados\\
    Entrega e Integração Contínua\\
    Testes Automatizados\\
    DevOps (AWS e Containers)\\
    Práticas/Metodologias Ágeis\\
\end{multicols}

\section{Algumas outras coisas que conheço:}
\begin{multicols}{3}
    \textbf{Técnicas:}\\
    APIs\\
    Database\\
    \textbf{Ferramentas:}\\
    Swagger\\
    DBeaver\\
    GIT\\
    IntelliJ IDEA\\
    VI\\
    Gradle\\
    Jenkins\\
    CircleCI\\
    Puppet\\
    Linux\\
    Spock\\
    Ansible\\
    \textbf{Plataformas:}\\
    Amazon Web Services\\
    Docker\\
    Heroku\\
    Tomcat\\
    \textbf{Frameworks:}\\
    AngularJS\\
    DurandalJS\\
    Grails\\
    Spring
\end{multicols}

\section{Outras coisas que gostaria de mencionar:}

Antes de começar na FITec, trabalhei em uma companhia customizando um ERP em java. Também trabalhei como estagiário no Centro de Informações da faculdade onde me graduei, ajudando a criar uma aplicação para suportar as necessidades dos estudantes. Também durante a faculdade, trabalhei em uma pesquisa relacionada com a indústria de cimento, ajudando na implementação de uma aplicação \textit{desktop} para otimização de processos.\\

Sempre tento particpar de eventos, workshops e cursos online relacionados à área para me atualizar. \\

Consigo compartilhar conhecimento para outras pessoas através de apresentações.\\

\end{document}
